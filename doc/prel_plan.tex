\documentclass[a4paper]{article}

\usepackage[swedish]{babel}
\usepackage[a4paper]{geometry}
\usepackage[utf8]{inputenc}
\usepackage{multicol}

\linespread{1.15}

\title{Preliminär planering}
\author{Philip Arvidsson}
\date{October 2015}

\begin{document}
    \maketitle

    \section{Inledning}
    Projektet delas in i tre iterationer á två veckor. I detta tidiga skede
    lämnas iteration två och tre därhän, och fokus ligger istället på den
    första iterationen.

    Vi inleder arbetet ett med en överenskommelse om systemets uppbyggnad,
    en övergripande arkitekturell design m.m. Under varje iteration förs
    protokoll i form av möten, problem som uppstår, vilka gruppmedlemmar som
    gjort vad etc.

    \section{Den första iterationen (27 okt. - 9 nov.)}

    I den första iterationen fokuserar vi på att skapa en preliminär planering
    (detta dokument), skapa en domänmodell, påbörja några use-cases samt bygga
    en icke-funktionell prototyp.

    De use-cases vi fokuserar på i början av iteration ett är följande:

    \begin{enumerate}
        \item Login \textit{(Player)}
        \item Logout \textit{(Player)}
        \item Handle account \textit{(User)}
        \item Play game \textit{(Player)}
    \end{enumerate}

    För att komma igång med arbetet på ett bra sätt kommer iteration ett bestå
    av flera ``mikroiterationer'' där vi delvis bygger upp det som krävs för att
    slutföra iterationen.

    Vid iterationens slut sammanställs alla dokument m.m. och lämnas in.
\end{document}
