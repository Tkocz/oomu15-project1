\documentclass[a4paper]{article}

\usepackage[swedish]{babel}
\usepackage[a4paper]{geometry}
\usepackage[utf8]{inputenc}
\usepackage{multicol}

\linespread{1.15}

\title{Preliminär planering}
\author{Philip Arvidsson}
\date{November 2015}

\begin{document}
    \maketitle

    \section{Inledning}
    Projektet delas in i tre iterationer á två veckor. I detta tidiga skede
    lämnas iteration två och tre därhän, och fokus ligger istället på den
    första iterationen.

    Vi inleder arbetet med en överenskommelse om systemets uppbyggnad, en
    övergripande arkitekturell design m.m. Under varje iteration förs protokoll
    i form av möten, problem som uppstår, vilka gruppmedlemmar som gort vad etc.

    \section{Den första iterationen (27 okt. - 9 nov.)}

    I den första iterationen fokuserar vi på att skapa en preliminär planering
    (detta dokument), skapa en domänmodell, påbörja några use-cases samt bygga
    en icke-funktionell prototyp.

    De use-cases vi fokuserar på i början av iteration ett är följande:

    \begin{enumerate}
        \item Login \textit{(Player)}
        \item Logout \textit{(Player)}
        \item Handle account \textit{(User)}
        \item Play game \textit{(Player)}
    \end{enumerate}

    För att komma igång med arbetet på ett bra sätt kommer iteration ett bestå
    av flera ``mikroiterationer'' där vi delvis bygger upp det som krävs för att
    slutföra iterationen. Vid iterationens slut sammanställs alla dokument m.m.
    och lämnas in.

    Iterationens arbete delas i övrigt upp i tre faser - analys, design och
    implementation - där analysfasen står för problemuppstälningen, design för
    problemlösningen och implementation för omsättning till pratik i form av ett
    körbart program.

    \vspace{0.5cm}

    \textbf{Beräknad tid för de olika faserna:}

    \begin{enumerate}
        \item \textbf{Analys} - 3 dagar (27/10 - 29/10)

              \textit{Under den inledande analysen ligger fokus på att förstå
                      problemuppställningen, varför analysen inte kommer bli
                      särskilt djuplodande.}

        \item \textbf{Design} - 4 dagar (2/11 - 4/11)

              \textit{Den första iterationen resulterar i en preliminär design
                      som lägger systemets grund utan detaljbeskrivningar.}

        \item \textbf{Implementation} - 4 dagar (5/11-9/11)

              \textit{Implementationen inleds och en prototyp skapas.}
    \end{enumerate}

    \section{Potentiella hinder}

    Det är svårt att förutse sammansättningen av use cases, sekvensdiagram m.m.
    varför vi möjligtvis kommer få ``glida'' fram och tillbaka mellan faserna
    något för att revidera detaljer.

    \section{Noteringar}

    \textbf{02/11 -15} \textit{Planering ändrad eftersom KAJA flyttat deadline
                               för inlämning en gång till.}
\end{document}
