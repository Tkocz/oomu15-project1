\documentclass[a4paper]{article}

\usepackage[swedish]{babel}
\usepackage[a4paper]{geometry}
\usepackage[utf8]{inputenc}
\usepackage{multicol}

\linespread{1.15}

\title{Planering för iteration 3}
\author{Philip Arvidsson}
\date{November 2015}

\begin{document}
    \maketitle

    \section{Planering}
    Nu jäklar ska det bli ordning och reda. Iteration 3 har följande att bjuda
    på (om allt går som tänkt, vilket det ju i och för sig sällan gör):

    \section{Analys}

    \begin{tabular}{ll}
        Play game & Färdigställd \\
        Handle advertisements & Färdigställd \\
        Webcam chat & Möjligtvis \\
    \end{tabular}

    \section{Design}
    Det finns inte mycket att säga om designen i iteration tre; den kommer att
    grunda sig i nuvarande arkitekturella design. Fokus kommer att ligga på
    multiplayer-funktionalitet.

    \section{Implementation}
    Vi ska implementera allt som går. Steam - släng er i väggen! Mer realistiskt
    bör vi få ihop ett fungerande multiplayerspel och hantering av annonser.

\end{document}
